\documentclass{beamer}
%
% Choose how your presentation looks.
%
% For more themes, color themes and font themes, see:
% http://deic.uab.es/~iblanes/beamer_gallery/index_by_theme.html
%
\mode<presentation>
{
  \usetheme{default}      % or try Darmstadt, Madrid, Warsaw, ...
  \usecolortheme{default} % or try albatross, beaver, crane, ...
  \usefonttheme{default}  % or try serif, structurebold, ...
  \setbeamertemplate{navigation symbols}{}
  \setbeamertemplate{caption}[numbered]
} 

\usepackage{polski}
\usepackage{listings}
\lstset{language=Matlab}          % Set your language (you can change the language for each code-block optionally)

\usepackage{sansmathaccent}
\pdfmapfile{+sansmathaccent.map}

\usepackage[utf8x]{inputenc}

\title[Image Processing Toolbox]{Przegląd algorytmów i narzędzi do analizy obrazów w Matlabie}
\author{Rafał \textsc{Grabiański} i Zbigniew \textsc{Królikowski}}
\institute{}
\date{29.05.2015}

\begin{document}

\begin{frame}
  \titlepage
\end{frame}

% Uncomment these lines for an automatically generated outline.
%\begin{frame}{Outline}
%  \tableofcontents
%\end{frame}

\section{Struktury danych dla obrazu}

\begin{frame}{Struktura danych dla obrazu}

\begin{block}{Reprezentacja obrazu}
Naturalną strukturą do przechowywania obrazów jest macierz. Mamy cztery typy reprezentacji, które
wpływają na to jak wartości w tej macierzy są przez program interpretowane.
\end{block}


\begin{itemize}
  \item \texttt{Binary (bilevel image)} - tablica wartości logicznych 1 i 0. Interpretowanych jako kolor czarny i biały.
  \item \texttt{Indexed (pseudocolor image)} - tablica wartości całkowitych, które są indeksami w mapie kolorów $m \times 3$.
  \item \texttt{Grayscale (intensity image)} - tablica wartości stanowiących kontrast danego piksela. Najczęściej do obrazów w skali szarości, chociaż można też mapować z colormapą.
  \item \texttt{Truecolor (RGB image)} - macierz $m \times n \times 3$, w której w trzecim wymiarze przechowujemy dokładną informację o kolorach czerwonym, zielonym i niebieskim.
\end{itemize}

\end{frame}

\begin{frame}[fragile]
\frametitle{Struktura danych dla obrazu}

\begin{example}
\begin{lstlisting}
RGB = reshape(ones(64,1)
	*reshape(jet(64),1,192),[64,64,3]);
R = RGB(:,:,1);
G = RGB(:,:,2);
B = RGB(:,:,3);
figure,	imshow(R)
figure, imshow(G)
figure, imshow(B)
figure, imshow(RGB)
\end{lstlisting}
\end{example}

\end{frame}

\section{Procedury wczytujące i zapisujące obrazy}

\begin{frame}[fragile]
\frametitle{Wczytywanie obrazu}
\begin{block}{Funkcja imread}
Wczytywanie obrazu do macierzy w matlabie sprowadza się do wywołania funkcji imread z nazwą obrazu. Po zawartości pliku imread rozpoznaje jego typ i zwraca macierz/-e obrazu. Można oczywiście w odpowiednim parametrze wyspecyfikować ten typ.
\end{block}

\begin{example}
\begin{lstlisting}
RGB = imread('football.jpg');
[X,map] = imread('trees.tif');
\end{lstlisting}
\end{example}

\end{frame}

\begin{frame}[fragile]
\frametitle{Wczytywanie obrazu - przykłady}
\begin{example}
\begin{lstlisting}
A = imread('ngc6543a.jpg');
image(A);
A(300, 300, 1)
A(300, 300, 2)
A(300, 300, 3)

[X,map] = imread('corn.tif');
if ~isempty(map)
    Im = ind2rgb(X,map);
end
whos Im;
\end{lstlisting}
\end{example}

\end{frame}

\begin{frame}[fragile]
\frametitle{Zapisywanie obrazu}
\begin{block}{Funkcja imwrite}
Funkcja imwrite zapisuje podaną macierz w postaci obrazu, którego typ rozpoznaje z podanej nazwy pliku albo przy użyciu odpowiedniego parametru.
\end{block}

\begin{example}
\begin{lstlisting}
imwrite(X,map,'clown.bmp')
imwrite(X,map,'clown.jpg')
\end{lstlisting}
\end{example}

\begin{alertblock}{Zadanie}

\begin{lstlisting}
X/100
imwrite(X/100, 'clown3.jpg')
\end{lstlisting}
\end{alertblock}

\end{frame}

\begin{frame}[fragile]
\frametitle{Konwertowanie pomiędzy formatami obrazów}
\begin{block}{Konwersja}
Nie ma specjalnych funkcji konwertujących pomiędzy różnymi formatami. Używamy już omówionych imread i imwrite do odpowiednich konwersji.
\end{block}
\end{frame}

\begin{frame}[fragile]
\frametitle{Pozyskiwanie informacji o pliku}
\begin{block}{Funkcja imfinfo}
Do odczytania typu obrazu i jego różnych właściwości możemy użyć funkcji imfinfo, w argumencie podając nazwę pliku.
\end{block}

\begin{example}
\begin{lstlisting}
info = imfinfo('ngc6543a.jpg')
\end{lstlisting}
\end{example}
\end{frame}

\section{Wyświetlanie i badanie obrazów}

\begin{frame}[fragile]
\frametitle{Wyświetlanie i badanie obrazów}
\begin{block}{Display and exploration tools}
Funkcje i aplikacje do wyświetlania i badania obrazów służą do oglądania zdjęć, pozyskiwania szczegółowych informacji o posczególnych pikselach, ustawiania kontrastu czy też pomiaru odległości. Wprowadzimy w tym referacie najbardziej podstawowe narzędzia.
\end{block}

\begin{block}{Funkcja imshow}
Funkcja imshow służy do otwierania w oknie Matlaba zdjęcia. Jeżeli nie mieści się ono na ekranie jest przeskalowywane. Standardowo po wyświetleniu jeden piksel na ekranie - jeden piksel zdjęcia.
\end{block}

\begin{example}
\begin{lstlisting}
moon = imread('moon.tif');
imshow(moon);
figure, imshow(moon, 'Border', 'tight');
\end{lstlisting}
\end{example}
\end{frame}

\begin{frame}[fragile]
\frametitle{Wyświetlanie i badanie obrazów - przykłady}
\begin{block}{Przykłady}
Prezentujemy jeszcze przykład wyświetlenia dwóch obrazów w jednej figurze. Subplot dzieli figurę, a subimage działa jak imshow, z tą jednak różnicą, że przed wyświetleniem konwertuje obraz na format truecolor. Przy wyświetlaniu zdjęć w jednej figurze musiałyby mieć w przeciwnym wypadku tę samą colormapę.
\end{block}

\begin{example}
\begin{lstlisting}
[X1,map1]=imread('forest.tif');
[X2,map2]=imread('trees.tif');
subplot(1,2,1), subimage(X1,map1)
subplot(1,2,2), subimage(X2,map2)
\end{lstlisting}
\end{example}

\end{frame}

\begin{frame}[fragile]
\frametitle{Image Viewer App}
\begin{block}{Image Viewer App}
Image Viewer App to prosty tool do wyświetlania obrazów i wykonywania podstawowych operacji na obrazach. To zintegrowane środowisko zawiera w sobie wiele pomocnych tooli:
\end{block}

\begin{itemize}
  \item \texttt{Pixel Information Tool} - do uzyskiwania informacji o bieżącym pikselu.
  \item \texttt{Pixel Region Tool} - do uzyskiwania informacji o grupie pikseli.
  \item \texttt{Distance Tool} - do pomiaru odległości między dwoma pikselami.
  \item \texttt{Adjust Contrast Tool} - do zmiany kontrastu wyświeltanego obrazu i modyfikowania danych.
  \item \texttt{Crop Image Tool} - do przycinania.
  \item \texttt{Display Range Tool} - do zmiany zakresu natężeń obrazu.
\end{itemize}
\end{frame}

\begin{frame}[fragile]
\frametitle{Image Viewer App - użycie}
\begin{block}{Funkcja imtool}
Aby uruchomić Image Viewer App, wybieramy odpowiednią opcję z poziomu menu File albo używamy funkcji imtool.
\end{block}

\begin{example}
\begin{lstlisting}
imtool('moon.tif');
imtool('moon.tif', 'InitialMagnification', 150)
\end{lstlisting}
\end{example}
\end{frame}

\section{Transformacje obrazów}

\begin{frame}
\frametitle{Transformacje geometryczne obrazu, spatial referencing, rejestracja obrazu}
\begin{block}{Plan}
Zajmiemy się na początku prostymi przekształceniami geometrycznymi obrazu by później skończyć na bardziej zaawansowanym przykładzie rejestracji obrazu.
\end{block}
\end{frame}

\begin{frame}[fragile]
\frametitle{Skalowanie obrazu}
\begin{block}{Funkcja imresize}
Aby zmienić rozmiar obrazu używamy funkcji imresize. Kluczowy jest magnification factor, który podajemy jako argument. Dla wartości większych od 1 obraz wyjściowy będzie większy niż oryginalny.
\end{block}

\begin{example}
\begin{lstlisting}
I = imread('circuit.tif');
J = imresize(I,1.25);
imshow(I)
figure, imshow(J)
\end{lstlisting}
\end{example}

\end{frame}

\begin{frame}[fragile]
\frametitle{Skalowanie obrazu - dodatkowe parametry}

\begin{block}{Interpolacja przy powiększaniu obrazu}
Do funkcji imresize możemy podać w apostrofach argument mówiący o metodzie interpolowania pikseli powstałych przy powięszeniu obrazu. Opcja 'bicubic' jest defaultowa.
\end{block}

\begin{block}{Zjawisko aliasingu podczas zmniejszania rozmiaru obrazu}
Domyślnie podczas zmniejszania rozmiaru obrazu, funkcja imresize używa antialiasingu. Możemy tę opcję wyłączyć dodając argumenty: 'Antialiasing', false.
\end{block}

\begin{example}
\begin{lstlisting}
I = imread('rice.png');
J = imresize(I, 0.5, 'Antialiasing', false);
figure, imshow(I), figure, imshow(J)
J = imresize(I, 0.5);
figure, imshow(J)
\end{lstlisting}
\end{example}

\end{frame}

\begin{frame}[fragile]
\frametitle{Obracanie obrazu}

\begin{block}{Funkcja imrotate}
Do obracania obrazu używamy funkcji imrotate podając w argumencie oprócz macierzy obrazu, kąt obrócenia. Dodatni kąt obraca w ruchu przeciwnym do ruchu wskazówek zegara.
\end{block}

\begin{example}
\begin{lstlisting}
I = imread('circuit.tif');
J = imrotate(I,35,'bilinear');
imshow(I)
figure, imshow(J)
\end{lstlisting}
\end{example}

\end{frame}


\begin{frame}[fragile]
\frametitle{Przycinanie obrazu}
\begin{block}{Funkcja imcrop}
Do przycinania obrazu używamy funkcji imcrop. Możemy wywołać ją interaktywnie lub podając konretne parametry w postaci [xmin, ymin, width, height].
\end{block}

\begin{example}
\begin{lstlisting}
I = imread('circuit.tif');
J = imcrop(I,[60 40 100 90]);
\end{lstlisting}
\end{example}

\end{frame}

\begin{frame}
\frametitle{Geometryczna transformacja 2D}

\begin{block}{Transformacja geometryczna}
Wykonanie transformacji w Image Processing Toolbox składa się z dwóch etapów.
\end{block}

\begin{itemize}
	\item Stworzenie obiektu transformacji \texttt{geometric transformation object}
	\item Użycie funkcji \texttt{imwarp} podając obraz do transformacji i obiekt transformacji
\end{itemize}
\end{frame}

\begin{frame}
\frametitle{Geometryczna transformacja 2D}
\begin{block}{Macierze transformacji}
Macierze transformacji służą temu by stosować rachunek macierzowy przy składaniu przekształcenia. Każdy piksel reprezentuje się przy pomocy macierzy $3 \times 1$ postaci [x, y, 1]. Przekształcenie to macierz, która po pomnożeniu przez punkt wskaże pozycję tego punktu. I tak macierz translacji o współrzędne $[t_x, t_y]$ będzie miała postać: \\
\centering
$T = \begin{bmatrix}
	1 & 0 & 0\\
	0 & 1 & 0\\
	t_x & t_y & 1
\end{bmatrix}$
\end{block}
\end{frame}

\begin{frame}
\frametitle{Inne przykłady macierzy transformacji}
\begin{block}{Skalowanie}
\centering
$S = \begin{bmatrix}
	s_x & 0 & 0\\
	0 & s_y & 0\\
	0 & 0 & 1
\end{bmatrix}$
\end{block}
\begin{block}{Rotacja}
\centering
$R = \begin{bmatrix}
	cos(\alpha) & sin(\alpha) & 0\\
	-sin(\alpha) & cos(\alpha) & 0\\
	0 & 0 & 1
\end{bmatrix}$
\end{block}
\end{frame}

\begin{frame}[fragile]
\frametitle{Przykład prostej translacji obrazu}
\begin{block}{Translacja 2D obrazu}
Najpierw odczytujemy obraz, który użyjemy do przekształcenia. Następnie tworzymy z macierzy transformacji obiekt transormacji funkcją affine2d. Ostatnim krokiem jest użycie funkcji imwarp podając obraz i uzyskany wcześniej obiekt.
Wszystko co zaraz zrobimy można zrobić też jedną funkcją imtranslate.
\end{block}

\end{frame}

\begin{frame}[fragile]
\frametitle{Przykład prostej translacji obrazu}
\begin{example}
\begin{lstlisting}
cb = checkerboard;
tform = affine2d([1 0 0; 0 1 0; 20 20 1]);
Rcb = imref2d(size(cb))
Rout = Rcb;
Rout.XWorldLimits(2) = Rout.XWorldLimits(2)+20;
Rout.YWorldLimits(2) = Rout.YWorldLimits(2)+20;
[out,Rout] = imwarp(cb,tform,'OutputView',Rout);
figure, subplot(1,2,1);
imshow(cb,Rcb);
subplot(1,2,2);
imshow(out,Rout)
\end{lstlisting}
\end{example}
\end{frame}

\begin{frame}
\frametitle{Rejestracja obrazu}
\begin{block}{Rejestracja obrazu}
Rejestracja obrazu jest procesem przenoszenia kilku zbiorów danych do jednego układu współrzędnych.
Wykonamy przykładową rejestrację dwóch obrazów używając spatial referencing.
\end{block}
\end{frame}

\begin{frame}{Przykładowa rejestracja obrazu}

\begin{itemize}
	\item Odczytanie dwóch przykładowych obrazów
	\item Wczytujemy gotowe wspólne punkty dla obu obrazów
	\item Tworzymy obiekt transformacji typu projective
	\item Używamy funkcji imwarp z gotowym obiektem do wykonania transformacji na przestawionym obrazie
	\item Wyświetlamy obrazy w odpowiednim układzie
\end{itemize}

\end{frame}

\begin{frame}[fragile]
\frametitle{Przykładowa rejestracja obrazu}
\begin{example}
\begin{lstlisting}
fixed = imread('westconcordorthophoto.png');
moving = imread('westconcordaerial.png');
load westconcordpoints
tform = fitgeotrans(movingPoints, fixedPoints, 
			'projective');
registered = imwarp(moving, tform,
			'FillValues', 255);
figure, imshow(registered);
Rfixed = imref2d(size(fixed));
registered1 = imwarp(moving,tform,
			'FillValues', 255,
			'OutputView',Rfixed);
				
figure, imshowpair(fixed,registered1,'blend');
\end{lstlisting}
\end{example}

\end{frame}

\begin{frame}{Rejestracja obrazu}
\begin{block}
Oczywiście istnieje cały szereg funkcji i opcji umożliwiających wykonywanie automatycznych rejestracji. Jedną z nich jest imregister. Tu jednak nie będziemy ich omawiać. Zainteresowanych odsyłamy do dokumentacji Matlaba.
\end{block}
\end{frame}

\section{Zmiana kontrastu}

\begin{frame}[fragile]
\frametitle{Ustawianie kontrastu}
\begin{block}{Ustawianie kontrastu}
Techniki, które teraz zaprezentujemy mogą być opisanie wspólnie mianem technik retuszujących, które mają za zadanie "poprawić" w jakiś subiektywny sposób obraz. Pierwszą jaką zaprezentujemy to zwiększenie kontrastu.
\end{block}

\begin{example}
\begin{lstlisting}
I = dicomread('CT-MONO2-16-ankle.dcm');
imtool(I,'DisplayRange',[])
\end{lstlisting}
\end{example}

\end{frame}

\begin{frame}{Ustawianie kontrastu}
\begin{block}{funkcja imadjust}
Funkcja imadjust zwiększa kontrast w taki sposób, że rozszerza skalę szarości do całego zakresu. Powoduje to, że obraz staje się wyraźniejszy i wykorzystywane są wszystkie odcienie szarości.
\end{block}
\end{frame}

\begin{frame}[fragile]
\frametitle{Ustawianie kontrastu}
\begin{example}
\begin{lstlisting}
I = imread('pout.tif');
imshow(I);
imtool(I);
imtool(J);
J = imadjust(I);
figure, imshow(J);
\end{lstlisting}
\end{example}
\end{frame}

\begin{frame}[fragile]
\frametitle{Ustawianie kontrastu}
\begin{block}{Funkcja imadjust}
Zwykłe użycie imadjust - czy to z ograniczeniem czy na całym zakrasie powoduje, że wartości rozciągają się liniowo na zadanym przedziale, jeślibyśmy jednak chcieli by było inaczej, musimy jako 4. argument podać współczynnik gamma.
\end{block}
\begin{example}
\begin{lstlisting}
[X,map] = imread('forest.tif');
I = ind2gray(X,map);
J = imadjust(I,[],[],0.5);
K = imadjust(I,[],[],2);
imshow(I)
%figure, imshow(J)
%figure, imshow(K)
imtool(J)
imtool(K)
\end{lstlisting}
\end{example}
\end{frame}

\begin{frame}[fragile]
\frametitle{Histogram equalization}
\begin{example}
\begin{lstlisting}
I = imread('pout.tif');
J = histeq(I);
imshow(J)
figure, imhist(J,64)
\end{lstlisting}
\end{example}
\end{frame}

\section{Filtrowanie obrazu}

\begin{frame}[fragile]
\frametitle{Filtrowanie obrazu}
\begin{block}{Funkcjia imfilter}
Filtrowaniu obrazu nie poświęcimy dużo czasu. Pokażemy tylkko kilka z wielu różnych możliwości:
\end{block}

\begin{example}
\begin{lstlisting}
originalRGB = imread('peppers.png');
imshow(originalRGB)
h = fspecial('motion', 50, 45);
filteredRGB = imfilter(originalRGB, h);
figure, imshow(filteredRGB)
boundaryReplicateRGB = imfilter(originalRGB, h, 'replicate');
figure, imshow(boundaryReplicateRGB)
\end{lstlisting}
\end{example}

\end{frame}

\begin{frame}{Filtrowanie obrazu}
\begin{block}{Filtrowanie obrazu}
Filtrowanie obrazu to proces, w którym wartość dla danego piksela jest ustalana na podstawie wartości z funkcji biorącej jako argumenty wartości dla pikseli znajdujących się w sąsiedztwie. Jeżeli jest to kombinacja liniowa wartości wtedy filtrowanie nazywamy filtrowaniem liniowym.
Omówimy dwa podstawowe filtrowania:
\end{block}

\begin{itemize}
	\item Convolution - liniowe filtrowanie z wagami podanymi w postaci macierzy (wyjaśnimy na przykładzie)
	\item Correlation - właściwie to samo co konwolucja, jedyna różnica, to że kernel korelacji nie jest obracany
\end{itemize}

\end{frame}

\begin{frame}[fragile]
\frametitle{Użycie gotowego filtra}
\begin{block}{Użycie filtra}
Użyjemy teraz predefiniowanego filtra na jednym z obrazów.
\end{block}

\begin{example}
\begin{lstlisting}
I = imread('moon.tif');
h = fspecial('?')
I2 = imfilter(I,h);
imshow(I)
figure
imshow(I2)
\end{lstlisting}
\end{example}

\begin{alertblock}{Zadanie}
Wiedząc że: h = $\begin{bmatrix}
   -0.1667  & -0.6667 &  -0.1667\\
   -0.6667  &  4.3333 &  -0.6667\\
   -0.1667  & -0.6667 &  -0.1667
\end{bmatrix}$\\
Powiedz jaki efekt uzyskamy na Księżycu używając tego filtra.
\end{alertblock}

\end{frame}

\section{Arytmetyka obrazów}


\begin{frame}[fragile]
\frametitle{Arytmetyka na obrazach}
\begin{block}{Wprowadzenie}
Arytmetyka na obrazach to po prostu operacje jakie znamy: dodawanie, odejmowanie, mnożenie dzielenie, tyle, że wykonywane na macierzach będącymi obrazami.
\end{block}
\end{frame}


\end{document}
